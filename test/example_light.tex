\documentclass[12pt]{article}

% \def\hideSolution{} % Execute this command to hide solution.

% Version: 20220413.1 %
\usepackage{amsmath}      % Math Symbols
\usepackage{amssymb}      % More Math Symbols
\usepackage{amsfonts}     % \mathbf{} etc. Math Fonts
\usepackage{amsthm}       % Theorem, Definition and Proof Environment
\usepackage{xparse}       % Optional Arguements in Command
\usepackage{graphicx}     % Insert Picture in LaTeX
\usepackage{subcaption}   % Image caption
\usepackage{placeins}     % Image placing
\usepackage{listings}     % Code snippets
\usepackage{mathtools}    % General tools for defining math symbol
\usepackage{physics}      % Beautiful Differentiation sign (dx)  
\usepackage{xcolor}       % Chromatize Texts
\usepackage{esvect}       % Beautiful Vector Sign Arrows
\usepackage{setspace}     % Set line spacing in theorem env
\usepackage{tikz}         % Plotting
\usepackage{soul}         % Highlighting
% \usepackage{fontspec}
\usepackage[framemethod=tikz]{mdframed} % Environment Highlight
% Some black magic
\makeatletter
\def\th@plain{%
  \thm@notefont{}% same as heading font
  \itshape % body font
}
\def\th@definition{%
  \thm@notefont{}% same as heading font
  \normalfont % body font
}
\makeatother
%

\theoremstyle{definition}
\newtheorem{theoremx}{Theorem}[section]
\newtheorem*{propositionx}{Proposition}
\newtheorem*{definition}{Definition}
\newtheorem*{solutionx}{Solution}

\theoremstyle{plain}
\newtheorem{corollary}{Corollary}[theoremx]
\newtheorem{lemma}[theoremx]{Lemma}
\newtheorem*{examplex}{Example}

% Define the \begin{note} style
\mdfdefinestyle{ExampleFrameStyle}{
  hidealllines=true,
  leftline=true,
  linecolor=\exampleBorderColor,
  middlelinewidth=.3em,
  backgroundcolor=\exampleBackgroundColor
}
\newenvironment{example}
{
  \begin{mdframed}[style=ExampleFrameStyle]
  \color{\exampleTextColor}
  \begin{examplex}
}
{
  \end{examplex}
  \end{mdframed}
}

\newenvironment{solution}
  { 
    \ifdefined\hideSolution
      \color{carnegieRed}THE SOLUTION IS HIDDEN UNDER CURRENT COMPILE SETTING
    \fi
    \color{\solutionTextColor}
    \pushQED{\qed}
    \renewcommand{\qedsymbol}{
      \emph{$\rightarrow$ End of Solution}
    }
    \solutionx
  }
  {
    \popQED
    \endsolutionx
  }

\newenvironment{proposition}[1][]
{\color{\propositionTextColor}\begin{propositionx}[#1]}
{\end{propositionx}}

% Define the \begin{theorem} style
\newcounter{theo}[section]\setcounter{theo}{0}
\renewcommand{\thetheo}{\arabic{section}.\arabic{theo}}

\newenvironment{theorem}[1][]{%
  \refstepcounter{theo}%
  \par
  \addvspace{5pt}
  \ifstrempty{#1}%
  {
    \mdfsetup{%
      frametitle={
        \tikz[baseline=(current bounding box.east),outer sep=0pt]
        \node[anchor=east,rectangle,fill=\theoremBackgroundColor]
        {\strut \color{\theoremStyleColor}Theorem~\thetheo};
  }}}
  {\mdfsetup{
    frametitle={
      \tikz[baseline=(current bounding box.east),outer sep=0pt]
      \node[anchor=east,rectangle, fill=\theoremBackgroundColor] %draw=\theoremStyleColor, fill=white, thick]
      {\strut \color{\theoremStyleColor}Theorem~\thetheo:~#1};
  }}}
  \mdfsetup{
    innertopmargin=5pt,
    innerbottommargin=5pt,
    linecolor=\theoremStyleColor,
    linewidth=1pt,
    topline=true,
    backgroundcolor=\theoremBackgroundColor,
    frametitleaboveskip=\dimexpr-\ht\strutbox\relax
  }
  \begin{mdframed}[nobreak=true]\color{\theoremTextColor}\relax}
{\end{mdframed}}


% Define the \begin{note} style
\mdfdefinestyle{NoteFrameStyle}{
  hidealllines=true,
  leftline=true,
  backgroundcolor=\noteBackgroundColor,
  linecolor=\noteBorderColor,
  middlelinewidth=.3em
}
\theoremstyle{remark}
\newtheorem*{mdnote}{Notes}
\newenvironment{note}
  {\begin{mdframed}[style=NoteFrameStyle]\color{\noteTextColor}\begin{mdnote}}
  {\end{mdnote}\end{mdframed}}

%%%%%%%%%%%%%%%%%%%%%%%%%%%%  My shortcuts [structures]
%%%%% Image Macro
\DeclareCaptionLabelFormat{custom}{
  \color{\captionColor}\textbf{#1 #2.}
}
\DeclareCaptionLabelSeparator{custom}{}
\captionsetup{
  labelformat=custom,
  labelsep=custom
}
\NewDocumentCommand{\pic}{ O{\textwidth} O{} m }
{
  \begin{center}
    \begin{figure}[ht]
      \centering\includegraphics[width=#1]{assets/#3}
      
      \caption{\color{\captionColor}#2}
    \end{figure}
  \end{center}\FloatBarrier
}
%%%%%%%%%%%%%%%%%

\newenvironment{amatrix}[2]{\left(\begin{array}{@{}*{#1}{c}|{c}@{}*{#2}{c}}}{\end{array}\right)}
\NewDocumentCommand{\augmatrix}{ O{3} O{1} m }{\begin{amatrix}{#1}{#2} #3 \end{amatrix}}
\newcommand{\hr}[0]{\rule{\linewidth}{0.4pt}}
\newcommand{\transpose}[1]{#1^{\top}}
\renewcommand{\abs}[1]{\left|#1\right|}
% Line Spacing adjustable matrix
\makeatletter
\renewcommand*\env@matrix[1][\arraystretch]{%
  \edef\arraystretch{#1}%
  \hskip -\arraycolsep
  \let\@ifnextchar\new@ifnextchar
  \array{*\c@MaxMatrixCols c}}
\makeatother
%%%%%%%%%%%%%%%%%%%%%%%%%%%%  My Symbols
\newcommand{\R}[0]{\mathbb{R}}
\newcommand{\Z}[0]{\mathbb{Z}}
\newcommand{\dif}[1]{\dd{#1}}
\newcommand{\hessian}[0]{\mathbf{H}}
\newcommand{\vbreak}[0]{\\[20pt]}
\DeclareMathOperator{\spn}{span}
\DeclareMathOperator{\crl}{curl}

% Style Setting
% Setup OpenSans Font
% \usepackage[default,scale=0.95]{opensans}
% \usepackage[T1]{fontenc}
\usepackage[scaled]{helvet} % Use helvetica instead
\renewcommand\familydefault{\sfdefault} 
\usepackage[T1]{fontenc}

% Setup typographical styles
\usepackage{parskip}
\usepackage[utf8]{inputenc}
\usepackage[margin=1in]{geometry}
\setlength{\parskip}{2.5mm}

% Setup color board
\definecolor{brightYellow}{HTML}{FFE564}
\definecolor{carnegieRed}{HTML}{A6192E}
\definecolor{darkBlue}{HTML}{002D72}
\definecolor{darkSteelGray}{HTML}{888F8F}
\definecolor{lightSteelGray}{HTML}{DDDFDF}
\definecolor{NotesBackground}{HTML}{FFF7D2}
\definecolor{NotesBorder}{HTML}{C5B100}

% Actual Color Used
\ifdefined\importTheme
\else
  \newcommand{\pageBackgroundColor}{white}

  \newcommand{\exampleTextColor}{darkSteelGray}
  \newcommand{\exampleBorderColor}{lightSteelGray}
  \newcommand{\exampleBackgroundColor}{\pageBackgroundColor}

  \newcommand{\captionColor}{darkSteelGray}

  % Hide Solution in Review Mode
  \ifdefined\hideSolution
    \newcommand{\solutionTextColor}{\pageBackgroundColor}
  \else
    \newcommand{\solutionTextColor}{darkSteelGray}
  \fi

  \newcommand{\propositionTextColor}{darkBlue}

  \newcommand{\theoremBackgroundColor}{\pageBackgroundColor}
  \newcommand{\theoremStyleColor}{darkBlue}
  \newcommand{\theoremTextColor}{black}

  \newcommand{\noteBackgroundColor}{NotesBackground}
  \newcommand{\noteBorderColor}{NotesBorder}
  \newcommand{\noteTextColor}{black}

  \newcommand{\highlighterColor}{brightYellow}

  \newcommand{\sectionColor}{carnegieRed}
  \newcommand{\subsectionColor}{carnegieRed}
\fi
% Setup Section Color
\usepackage{sectsty}
\sectionfont{
  \color{\sectionColor}
}
\subsectionfont{
  \color{\subsectionColor}
}
\sethlcolor{\highlighterColor}

% Header Page Number
\usepackage{fancyhdr}
\fancyfoot{}
\rhead{\thepage}
\pagestyle{fancy}
\setlength{\headheight}{14.49998pt}

\usepackage{lipsum} % This package is used to generate random text

\begin{document}
    \tableofcontents

    \newpage

    \section{Font, Color, and Typography}
    \subsection{Example Texts}

    \textbf{We use Arial as font, which is much `modern' than default font}.

    (Random Texts as placeholder) \lipsum[66]    % Some Random Text

    \subsection{Highlight Texts}

    You can \hl{highlight the texts} using `hl' command. However, it is not perfect. When applying on inline-math, there will have some problems related with line-height. \hl{$\dfrac{1}{2}$ (Here's an example)}.

    \subsection{Shortcuts for Italic and Bold}

    Import the settings in `./.vscode/keybindings.json', you can select some words in LaTeX and press `cmd + shift + i' for `\textit{textit}' and `cmd + shift + b' for `\textbf{bold}'.

    \subsection{Images}

    Store the image file into the \textbf{assets} folder, and use \textbf{pic[imageWidth][caption]$\{$fileName$\}$} command to insert images conveniently.

    \pic[300pt][some caption here]{assets_1.pdf}

    By default (if you don't pass-in any value), the image width will be the text-width, namely the page width, and caption will be empty.

    \newpage
    \section{Environments}
    \subsection{Definition Environment}
    \begin{definition}[Example Definition]
        (Random Texts as placeholder) \lipsum[75]% Some Random Text
        
        \begin{equation*}
            a^2 + b^2 = c^2
        \end{equation*}
    \end{definition}

    \textit{\textbf{The argument here is optional, below is what happens when first argument of Definition environment is empty}}

    \begin{definition}[]
        (Random Texts as placeholder) \lipsum[75]% Some Random Text
        \begin{equation*}
            a^2 + b^2 = c^2
        \end{equation*}
    \end{definition}

    \subsection{Example and Solution Environments}

    \begin{example}
        "Example Example" here lol \dots

        Equations will be colored as well.
        \begin{equation*}
            \begin{aligned}
                &\begin{bmatrix}
                    p_0'(t) \\ p_1'(t) \\ p_2'(t) \\ \vdots
                \end{bmatrix} &= 
                &\begin{bmatrix}
                    -\lambda & \mu & 0 & 0 & \cdots\\
                    \lambda & -(\lambda + \mu) & \mu & 0 & \cdots\\
                    0 & \lambda & -(\lambda + \mu) & \mu & \cdots\\
                    \vdots & \vdots & \vdots & \vdots & \ddots
                \end{bmatrix}
                &\begin{bmatrix}
                    p_0(t) \\ p_1(t) \\ p_2(t) \\ \vdots
                \end{bmatrix}
            \end{aligned}
        \end{equation*}

    \end{example}

    \begin{solution}
        (Random Texts as placeholder) \lipsum[65] % Some Random Text

    \end{solution}

    \subsection{Theorem Environment}

    Below are more environments defined with beautiful style.

    \begin{theorem}[Example Theorem Here]
        (Random Texts as placeholder) \lipsum[66]
    \end{theorem}

    Some arguements are optional here \dots

    \begin{theorem}[]
        (Random Texts as placeholder) \lipsum[75]
    \end{theorem}

    \subsection{Note Environment}

    You can use `note' environment as an emphasize.

    \begin{note}
        Something that is really important can be emphasized here!
    \end{note}

    \subsection{Proposition Environment}
    \begin{proposition}
        This is an example proposition \dots
    \end{proposition}

    \subsection{Proof Environment}

    \begin{proof}
        And here's the proof to example proposition!

        The little square of Q.E.D. is added automatically at the end of proof environment.

    \end{proof}

    \newpage
    \section{Fancy Stuff}
    \subsection{Hide solutions}

    Un-comment the line $3$ of this file and recompile the tex, you will see that all contents in solution environment is 'hidden'. This is useful when reviewing the notes for exam.

    \subsection{Dark Theme}

    Import the `theme/dark.tex' \textbf{before} where you import `style-v1.tex' to activate the dark theme.

    See `example\_dark.tex' and `example\_dark.pdf' for example and result.

    \begin{note}
        In fact, you can write your own theme by mimicking the `dark.tex'!
    \end{note}

    \subsection{Git-ignore}

    Product (pdf) and by-product (aux, log, etc.) of LaTeX compilation is excluded from the version tracing by .gitignore file.

    \begin{note}
        When using pdf as a graphic, you need to add `assets' prefix before the actual file. In this case, the file will be tracked by Git.

        \begin{center}
        \begin{tabular}{c | c}
            File Name & Will be Tracked ?\\
            \hline
            1.pdf   &   \\
            assets\_1.pdf & $\checkmark$\\
            assets1.pdf & $\checkmark$\\
            asset\_1.pdf & \\
        \end{tabular}
        \end{center}
    \end{note}

    \subsection{MakeFile}

    Some basic commands are included in the makefile.

    \begin{itemize}
        \item \textbf{make} - will clean all the auxiliary files created by LaTeXmk, leaving only tex and pdf files.
        \item \textbf{make rerun} - will delete all the outputs and auxiliary files of LaTeXmk, and recompile every tex file in the directory.
    \end{itemize}
\end{document}