% Version: 20220413.1 %
\usepackage{amsmath}      % Math Symbols
\usepackage{amssymb}      % More Math Symbols
\usepackage{amsfonts}     % \mathbf{} etc. Math Fonts
\usepackage{amsthm}       % Theorem, Definition and Proof Environment
\usepackage{xparse}       % Optional Arguements in Command
\usepackage{graphicx}     % Insert Picture in LaTeX
\usepackage{subcaption}   % Image caption
\usepackage{placeins}     % Image placing
\usepackage{listings}     % Code snippets
\usepackage{mathtools}    % General tools for defining math symbol
\usepackage{physics}      % Beautiful Differentiation sign (dx)  
\usepackage{xcolor}       % Chromatize Texts
\usepackage{esvect}       % Beautiful Vector Sign Arrows
\usepackage{setspace}     % Set line spacing in theorem env
\usepackage{tikz}         % Plotting
\usepackage{soul}         % Highlighting
% \usepackage{fontspec}
\usepackage[framemethod=tikz]{mdframed} % Environment Highlight
% Some black magic
\makeatletter
\def\th@plain{%
  \thm@notefont{}% same as heading font
  \itshape % body font
}
\def\th@definition{%
  \thm@notefont{}% same as heading font
  \normalfont % body font
}
\makeatother
%

\theoremstyle{definition}
\newtheorem{theoremx}{Theorem}[section]
\newtheorem*{propositionx}{Proposition}
\newtheorem*{definition}{Definition}
\newtheorem*{solutionx}{Solution}

\theoremstyle{plain}
\newtheorem{corollary}{Corollary}[theoremx]
\newtheorem{lemma}[theoremx]{Lemma}
\newtheorem*{examplex}{Example}

% Define the \begin{note} style
\mdfdefinestyle{ExampleFrameStyle}{
  hidealllines=true,
  leftline=true,
  linecolor=\exampleBorderColor,
  middlelinewidth=.3em,
  backgroundcolor=\exampleBackgroundColor
}
\newenvironment{example}
{
  \begin{mdframed}[style=ExampleFrameStyle]
  \color{\exampleTextColor}
  \begin{examplex}
}
{
  \end{examplex}
  \end{mdframed}
}

\newenvironment{solution}
  { 
    \ifdefined\hideSolution
      \color{carnegieRed}THE SOLUTION IS HIDDEN UNDER CURRENT COMPILE SETTING
    \fi
    \color{\solutionTextColor}
    \pushQED{\qed}
    \renewcommand{\qedsymbol}{
      \emph{$\rightarrow$ End of Solution}
    }
    \solutionx
  }
  {
    \popQED
    \endsolutionx
  }

\newenvironment{proposition}[1][]
{\color{\propositionTextColor}\begin{propositionx}[#1]}
{\end{propositionx}}

% Define the \begin{theorem} style
\newcounter{theo}[section]\setcounter{theo}{0}
\renewcommand{\thetheo}{\arabic{section}.\arabic{theo}}

\newenvironment{theorem}[1][]{%
  \refstepcounter{theo}%
  \par
  \addvspace{5pt}
  \ifstrempty{#1}%
  {
    \mdfsetup{%
      frametitle={
        \tikz[baseline=(current bounding box.east),outer sep=0pt]
        \node[anchor=east,rectangle,fill=\theoremBackgroundColor]
        {\strut \color{\theoremStyleColor}Theorem~\thetheo};
  }}}
  {\mdfsetup{
    frametitle={
      \tikz[baseline=(current bounding box.east),outer sep=0pt]
      \node[anchor=east,rectangle, fill=\theoremBackgroundColor] %draw=\theoremStyleColor, fill=white, thick]
      {\strut \color{\theoremStyleColor}Theorem~\thetheo:~#1};
  }}}
  \mdfsetup{
    innertopmargin=5pt,
    innerbottommargin=5pt,
    linecolor=\theoremStyleColor,
    linewidth=1pt,
    topline=true,
    backgroundcolor=\theoremBackgroundColor,
    frametitleaboveskip=\dimexpr-\ht\strutbox\relax
  }
  \begin{mdframed}[nobreak=true]\color{\theoremTextColor}\relax}
{\end{mdframed}}


% Define the \begin{note} style
\mdfdefinestyle{NoteFrameStyle}{
  hidealllines=true,
  leftline=true,
  backgroundcolor=\noteBackgroundColor,
  linecolor=\noteBorderColor,
  middlelinewidth=.3em
}
\theoremstyle{remark}
\newtheorem*{mdnote}{Notes}
\newenvironment{note}
  {\begin{mdframed}[style=NoteFrameStyle]\color{\noteTextColor}\begin{mdnote}}
  {\end{mdnote}\end{mdframed}}

%%%%%%%%%%%%%%%%%%%%%%%%%%%%  My shortcuts [structures]
%%%%% Image Macro
\DeclareCaptionLabelFormat{custom}{
  \color{\captionColor}\textbf{#1 #2.}
}
\DeclareCaptionLabelSeparator{custom}{}
\captionsetup{
  labelformat=custom,
  labelsep=custom
}
\NewDocumentCommand{\pic}{ O{\textwidth} O{} m }
{
  \begin{center}
    \begin{figure}[ht]
      \centering\includegraphics[width=#1]{assets/#3}
      
      \caption{\color{\captionColor}#2}
    \end{figure}
  \end{center}\FloatBarrier
}
%%%%%%%%%%%%%%%%%

\newenvironment{amatrix}[2]{\left(\begin{array}{@{}*{#1}{c}|{c}@{}*{#2}{c}}}{\end{array}\right)}
\NewDocumentCommand{\augmatrix}{ O{3} O{1} m }{\begin{amatrix}{#1}{#2} #3 \end{amatrix}}
\newcommand{\hr}[0]{\rule{\linewidth}{0.4pt}}
\newcommand{\transpose}[1]{#1^{\top}}
\renewcommand{\abs}[1]{\left|#1\right|}
% Line Spacing adjustable matrix
\makeatletter
\renewcommand*\env@matrix[1][\arraystretch]{%
  \edef\arraystretch{#1}%
  \hskip -\arraycolsep
  \let\@ifnextchar\new@ifnextchar
  \array{*\c@MaxMatrixCols c}}
\makeatother
%%%%%%%%%%%%%%%%%%%%%%%%%%%%  My Symbols
\newcommand{\R}[0]{\mathbb{R}}
\newcommand{\Z}[0]{\mathbb{Z}}
\newcommand{\dif}[1]{\dd{#1}}
\newcommand{\hessian}[0]{\mathbf{H}}
\newcommand{\vbreak}[0]{\\[20pt]}
\DeclareMathOperator{\spn}{span}
\DeclareMathOperator{\crl}{curl}
