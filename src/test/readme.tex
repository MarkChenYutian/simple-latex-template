\documentclass{article}

\usepackage[NoIndent]{../lib/cleanNote}

\usepackage{lipsum} % This package is used to generate random text

\title{\code{cleanNote} - a Fancy Note Package}
\author{Yutian Chen}

\begin{document}
    \maketitle
    \tableofcontents
    \newpage

    This package is an encapsulation of styling and environment definition for daily note-taking. As the name reveals, We aimed to make the output PDF more `modernized' and `clean'.

    \code{cleanNote} also provides some useful functions like \textbf{dark mode} and \textbf{hide solution} (for review use).

    \section{Theme and Options for \code{cleanNote}}

\begin{codeblock}[tex]
% Argument Syntax
\usepackage[Dark?, HideSolution?, NoIndent?]{cleanNote}

% Examples
\usepackage{cleanNote}                     % Normal setup (light theme)
\usepackage[Dark]{cleanNote}               % Dark Theme
\usepackage[Dark, HideSolution]{cleanNote} % Hide solution, Dark theme
\usepackage[NoIndent]{cleanNote}    % No indent at beginning of paragraph
\end{codeblock}

    Currently, \code{cleanNote} supports three options:

    \begin{itemize}
        \item \code{Dark} - Passing this argument into \code{cleanNote} will make output PDF set to "dark mode"
        \item \code{HideSolution} - Passing this argument into \code{cleanNote} will make output PDF hide all solution (by setting color of solution text same as page background).
        \item \code{NoIndent} - Passing this argument into \code{cleanNote} will remove the indentation at beginning of paragraph.
    \end{itemize}

    \section{Environments}

    Multiple environments are defined in \code{cleanNote}. This section will introduce and demonstrate each of them:

    \subsection{Definition}

\begin{codeblock}[tex]
\begin{definition}[<Term>]?
    <Content>
\end{definition}
\end{codeblock}

    \begin{definition}[Terminology]
        The technical or special terms used in a business, art, science, or special subject (\href{https://www.merriam-webster.com/dictionary/terminology}{source})
    \end{definition}

    Note that the argument \code{<term>} is optional, meaning that you can create a definition without passing any argument.

    \begin{definition}
        This is a definition without the \code{<term>} argument.
    \end{definition}

    \subsection{Theorem}

\begin{codeblock}[tex]
\begin{theorem}[<Theorem_name>]?
    <Content>
\end{theorem}
\end{codeblock}

\begin{theorem}[Pythagorean Theorem]
    If $c$ denotes the length of the hypotenuse and $a$ and $b$ denote the two lengths of the legs of a right triangle, then the Pythagorean theorem can be expressed as the Pythagorean equation: 
    \begin{equation*}
        a^2 + b^2 = c^2
    \end{equation*}
\end{theorem}

Similar to environment \code{definition}, the \code{theorem} environment's argument can be omitted.

\begin{theorem}[]
    This theorem is too trivial to receive a proper name :(
\end{theorem}

\subsection{Proposition}

\begin{codeblock}[tex]
\begin{proposition}[<Proposition_name>]?
    <Content>
\end{proposition}
\end{codeblock}

\begin{proposition}
    Proposition is important, but not so important as a theorem, so it does not have a fancy box.
\end{proposition}

\begin{proposition}[Some Name]
    Maybe some proposition is more important than others that it also deserves a name
\end{proposition}

\subsection{Proof}

\begin{codeblock}[tex]
\begin{proof}
    <Content>
\end{proof}
\end{codeblock}

    The \code{proof} environment is similar to the \code{proof} environment provided in \code{amsmath} package. An empty square represents Q.E.D. will be shown on the right at the end of proof.

    \begin{proof}
        Here's a simple proof that the Q.E.D. square does exist at the end of proof.

    \end{proof}

    \subsection{Note}

    Something is so important that it is better to be highlighted from normal texts. If that's the case, then \code{note} environment is the one you need!

\begin{codeblock}[tex]
\begin{note}
    <Content>
\end{note}
\end{codeblock}

    \begin{note}
        Something is just so special that it deserves a beautiful background.
    \end{note}

    \subsection{Example}

\begin{codeblock}[tex]
\begin{example}[<Question_name>]?
    <Content>
\end{example}
\end{codeblock}

\begin{example}[Example Question]
    Some `Example example' is happening here haha.
\end{example}

\begin{example}
    Example of \code{example} environment without argument.
\end{example}

    \subsection{Solution}

    Solution is just plain block with `End of solution' marked at the end of block. However, when \code{HideSolution} argument is passed into the \code{cleanNote} package, all contents in the solution package will be hidden. (unless you explicitly set a color for the text!)

    \begin{solution}
        Example Solution...I really like the beautiful typeset of math equations in LaTeX.
        \begin{equation*}
            \int_{\mathrm{Born}}^{\mathrm{Death}}{\mathrm{"My\;Heart\;is\;in\;the\;Work!"} \dd{t}}
        \end{equation*}
    \end{solution}

    \subsection{Code Block}

    \code{codeblock} environment can be applied whenever you want to insert a chunk of code in your PDF. It's syntax highlighting module (which relies on a python module called \code{pygmentize}) supports almost any language you can think of! (Sadly, no support for \code{C0} and \code{BC0}, though)

    \begin{note}
        Using code block feature with syntax highlighting requires a properly installed \code{pygmentize} package and also \code{--shell-escape} flag in compiler setting
    \end{note}

\begin{codeblock}[tex]
\begin {codeblock} [<Language>]?
<Code_content>
\end {codeblock}
\end{codeblock}

\begin{codeblock}[python]
def main():
    print("Hello world")        # Python is the best language :)
\end{codeblock}

\begin{codeblock}[javascript]
function greeting() {
    console.log("Hello world"); // I like adding ; at the end of JS!
}
\end{codeblock}

Maybe you are wondering what will happen if there's a suuuuuuper long line in your code, so here's a demo:

\begin{codeblock}[typescript]
const f = (UIHook: ReactHooks, mem: C0HeapAllocator, arg1: C0Value<Maybe<"value">>, arg2: C0Value<Maybe<"string">>) => { ... };
\end{codeblock}

\section{Integration with Visual Studio Code}
\subsection{\code{ltex} - Syntax Checking Plugin}

If you are using \code{ltex} plugin as a syntax checker for your LaTeX project, you can insert following configurations into \code{./.vscode/settings.json} file.

These configurations will tell \code{ltex} not to check syntax for contents in \code{Codeblock} environment or \code{\textbackslash code\textbraceleft...\textbraceright} macro.

\begin{codeblock}[json]
"ltex.latex.environments": {
    "lstlisting": "ignore",
    "verbatim": "ignore",
    "codeblock": "ignore",
},
"ltex.latex.commands": {
    "\\label{}": "ignore",
    "\\documentclass[]{}": "ignore",
    "\\cite{}": "dummy",
    "\\cite[]{}": "dummy",
    "\\code{}": "dummy"
}
\end{codeblock}

\subsection{Code Snippets}

You can also import the \code{tex\_snippet.code-snippets} into \code{./.vscode} directory to apply some simple snippets to your LaTeX project.

The available snippets are listed below:

\begin{enumerate}
    \item \code{EQ*} - Equation block
        \begin{codeblock}[tex]
\begin{equation*}
    <1>
\end{equation*}
        \end{codeblock}
    \item \code{AEQ*} - Multiline Equation Block with Alignment
\begin{codeblock}[tex]
\begin{equation*}
    \begin{aligned}
        <1>
    \end{aligned}
\end{equation*}
\end{codeblock}
    \item \code{EG} - Example Environment Block
\begin{codeblock}[tex]
\begin{example}
    <1>
\end{example}
\end{codeblock}
    \item \code{SOL} - Solution Environment Block
\begin{codeblock}[tex]
\begin{solution}
    <1>
\end{solution}
\end{codeblock}
    \item \code{CODE} - Code Block Environment
\begin{codeblock}[tex]
\begin {codeblock}[<1>]
<2>
\end {codeblock}
\end{codeblock}
    \item \code{LR[]} - Quick Left Right Square Bracket
\begin{codeblock}[tex]
\left [ <1> \right ]
\end{codeblock}
    \item \code{LR\textbraceleft\textbraceright} - Quick Left Right Bracket
\begin{codeblock}[tex]
\left\{ <1> \right\}
\end{codeblock}
    \item \code{DEF} - Definition Environment
\begin{codeblock}[tex]
\begin{definition}[<1>]
    <2>
\end{definition}
\end{codeblock}
    \item \code{RA} - Right arrow under Math Mode
\begin{codeblock}[tex]
\rightarrow
\end{codeblock}
    \item \code{THM} - Theorem Environment
\begin{codeblock}[tex]
\begin{theorem}[<1>]
    <2>
\end{theorem}
\end{codeblock}
    \item \code{VBAR} - Vertical bar within left-right brackets
\begin{codeblock}[tex]
\,\middle |\,
\end{codeblock}
    \item \code{TAB} - Quick framework for tabular environment
\begin{codeblock}[tex]
\begin{center}
\begin{tabular}{<1>}
    <2>
\end{tabular}
\end{center}
\end{codeblock}
    \item \code{FIGBLOCK} - Figure block (takes complete width on page)
\begin{codeblock}[tex]
\begin{figure}[ht]
   \begin{center}
   \centering\includegraphics[width=<1>]{assets/<2>}
   \caption{\color{\captionColor} <3>}
   \end{center}
\end{figure}
\end{codeblock}
    \item \code{FIGINLINE} - Figure inline (figure that floats on left / right, default right)
\begin{codeblock}[tex]
\begin{wrapfigure}{r}{<1>\textwidth}
    \centering\includegraphics[width=<2>\textwidth]{<3>}
\end{wrapfigure}
\end{codeblock}
\end{enumerate}

For the actual configuration files, see Section 4 Appendix.

\subsection{Key Bindings}

You can also import the key bindings file into \code{VSCode} setting (\code{keybindings.json}).

\begin{codeblock}[json]
[
    {
        "key": "ctrl+shift+b",
        "command": "editor.action.insertSnippet",
        "when": "editorHasSelection",
        "args": {
            "snippet": "\\textbf{${TM_SELECTED_TEXT}}$0"
        }
    },
    {
        "key": "ctrl+shift+i",
        "command": "editor.action.insertSnippet",
        "when": "editorHasSelection",
        "args": {
            "snippet": "\\textit{${TM_SELECTED_TEXT}}$0"
        }
    }
]
\end{codeblock}

After importing these configurations, when you select some text in editor and press \code{ctrl + shift + b}, the text will become \textbf{bold} automatically.

Similarly, pressing \code{ctrl + shift + i} will make text \textit{italic}.

\section{Appendix: Full Snippet Setup File}

\begin{codeblock}[json]
{
    "Clean Equation Block" : {
        "scope": "latex",
        "prefix": "EQ*",
        "body":[
            "\\begin{equation*}",
            "    $1",
            "\\end{equation*}"
        ]
    },
    "Aligned Equation Block": {
        "scope": "latex",
        "prefix": "AEQ*",
        "body": [
            "\\begin{equation*}",
            "    \\begin{aligned}",
            "        $1",
            "    \\end{aligned}",
            "\\end{equation*}"
        ]
    },
    "Example Block": {
        "scope": "latex",
        "prefix": "EG",
        "body":[
            "\\begin{example}",
            "    $1",
            "\\end{example}"
        ]
    },
    "Solution Block":{
        "scope": "latex",
        "prefix": "SOL",
        "body":[
            "\\begin{solution}",
            "    $1",
            "\\end{solution}"
        ]
    },
    "Code Block":{
        "scope": "latex",
        "prefix": "CODE",
        "body": [
            "\\begin{codeblock}[$1]",
            "$2",
            "\\end {codeblock}"
        ]
    },
    "Left Right []":{
        "scope": "latex",
        "prefix": "LR[]",
        "body":[
            "\\left[ $1 \\right]"
        ]
    },
    "Left Right {}":{
        "scope": "latex",
        "prefix": "LR{}",
        "body":[
            "\\left\\{ $1 \\right\\\\}"
        ]
    },
    "Definition Block": {
        "scope": "latex",
        "prefix": "DEF",
        "body":[
            "\\begin{definition}[$1]",
            "    $2",
            "\\end{definition}"
        ]
    },
    "Right Arrow":{
        "scope": "latex",
        "prefix": "RA",
        "body":[
            "\\rightarrow"
        ]
    },
    "Theorem Block": {
        "scope": "latex",
        "prefix": "THM",
        "body":[
            "\\begin{theorem}[$1]",
            "    $2",
            "\\end{theorem}"
        ]
    },
    "Vertical Bar": {
        "scope": "latex",
        "prefix": "VBAR",
        "body":[
            "\\,\\middle |\\,"
        ]
    },
    "Table": {
        "scope": "latex",
        "prefix": "TAB",
        "body":[
            "\\begin{center}",
            "\\begin{tabular}{$1}",
            "    $2",
            "\\end{tabular}",
            "\\end{center}"
        ],
    },
    "Block Figure": {
        "scope": "latex",
        "prefix": "FIGBLOCK",
        "body": [
            "\\begin{figure}[ht]",
            "   \\begin{center}",
            "   \\centering\\includegraphics[width=$1]{assets/$2}",
            "   \\caption{\\color{\\captionColor} $3}",
            "   \\end{center}",
            "\\end{figure}"
        ]
    },
    "Inline Figure": {
        "scope": "latex",
        "prefix": "FIGINLINE",
        "body": [
            "\\begin{wrapfigure}{r}{$1\\textwidth}",
            "    \\centering\\includegraphics[width=$2\\textwidth]{$3}",
            "\\end{wrapfigure}"
        ]
    }
}   
\end{codeblock}

\end{document}
